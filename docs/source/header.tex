%------------------------------------------------------------------
%------------------------------------------------------------------
%------------------------------------------------------------------
%------------------------------------------------------------------
{\global\def\putFig#1#2#3{
        \begin{figure}[tp]
           \begin{center}
              \epsfxsize=#3in
              \epsfbox{figures/#1.eps}
           \end{center}
           \caption{\small{#2}}
           \label{fig:#1}
        \end{figure}
        }
} {\global\def\putFrag#1#2#3#4{
        \begin{figure}[tp]
           \begin{center}
              #4
              \epsfxsize=#3in
              \epsfbox{figures/#1.eps}
           \end{center}
           \caption{\small{#2}}
           \label{fig:#1}
        \end{figure}
        }
} {\global\def\twoFig#1#2#3{
        \begin{figure}[tp]
           \begin{center}
              \hfill
              \epsfxsize=2.75in
              \epsfbox{figures/#1.eps}
              \hfill
              \epsfxsize=2.75in
              \epsfbox{figures/#2.eps}
           \end{center}
           \caption{\small{#3}}
           \label{fig:#1}
        \end{figure}
        }
} {\global\def\twoFrag#1#2#3#4{
        \begin{figure}[tp]
           \begin{center}
              #4
              \hfill
              \epsfxsize=2.75in
              \epsfbox{figures/#1.eps}
              \hfill
              \epsfxsize=2.75in
              \epsfbox{figures/#2.eps}
           \end{center}
           \caption{\small{#3}}
           \label{fig:#1}
        \end{figure}
        }
} {\global\def\tempFig#1#2#3{
        \begin{figure}[h]
           \begin{center}
              \framebox[6in]{
                \bigskip
                {\bf Figure will go here.}
                \bigskip
              }
           \end{center}
           \caption{\small{#2}}
           \label{fig:#1}
        \end{figure}
        }
}
%------------------------------------------------------------------

 % math shortcuts
 \newcommand{\defn}{:=}
 \newcommand{\limit}[2]{\lim_{#1 \rightarrow #2}}
 \newcommand{\tvec}[1]{\Tilde{{\mathbf{#1}}}}
 \newcommand{\bvec}[1]{\boldsymbol{#1}}
 \newcommand{\ovec}[1]{{\mathbf{\Bar{#1}}}}
 \renewcommand{\vec}[1]{{\boldsymbol{#1}}}
 \newcommand{\mat}[1]{\ensuremath{\begin{pmatrix}#1\end{pmatrix}}}
 \newcommand{\smallmat}[1]{\ensuremath{
        \begin{smallmatrix}#1\end{smallmatrix}}}
 %\newcommand{\inprod}[1]{\ensuremath{\langle #1 \rangle}}
 \newcommand{\ip}[1]{\hspace{1mm}<\hspace{-1mm}#1\hspace{-1mm}>}
 \newcommand{\norm}[1]{\ensuremath{\| #1 \|}}
 \newcommand{\mc}[1]{\ensuremath{\mathcal{#1}}}
 \newcommand{\st}{{\;|\;}}
 \newcommand{\barst}[1]{\ensuremath{\text{\raisebox{-0.5mm}{$\bigl|_{#1}$}}}}
 \newcommand{\Barst}[1]{\ensuremath{\text{\raisebox{-0.5mm}{$\biggl|_{#1}$}}}}
 \newcommand{\zci}{{\textstyle\frac{1}{z^*}}}
 \newcommand{\Real}{{\mathbb{R}}}
 \newcommand{\Complex}{{\mathbb{C}}}
 \newcommand{\Int}{{\mathbb{Z}}}
 \newcommand{\Rat}{{\mathbb{Q}}}
 \newcommand{\Nat}{{\mathbb{N}}}

 % operators
 \DeclareMathOperator{\real}{Re}
 \DeclareMathOperator{\imag}{Im}
 \DeclareMathOperator{\sgn}{sgn}
 \DeclareMathOperator{\csgn}{csgn}
 \DeclareMathOperator{\E}{E}
 \DeclareMathOperator{\kurt}{kurt}
 \DeclareMathOperator{\Kurt}{{\mc K}}
 \DeclareMathOperator{\row}{row}
 \DeclareMathOperator{\col}{col}
 \DeclareMathOperator{\spn}{span}
 \DeclareMathOperator{\tr}{tr}
 \DeclareMathOperator{\diag}{diag}

% % theorems
% \newtheorem{theorem}{Theorem}
% \newtheorem{lemma}{Lemma}
% \newtheorem{conjecture}{Conjecture}
% \newtheorem{corollary}{Corollary}

 % Theorems
 \newtheorem{theorem}{Theorem}
 \newtheorem{proposition}[theorem]{Proposition}%[chapter]
 \newtheorem{corollary}[theorem]{Corollary}%[chapter]
 \newtheorem{conjecture}[theorem]{Conjecture}%[chapter]
 \newtheorem{lemma}[theorem]{Lemma}%[chapter]
 \newtheorem{claim}[theorem]{Claim}
 \newtheorem{definition}[theorem]{Definition}%[chapter]
% \newenvironment{proof}{{\sl Proof\/}:\ \ }{\qed\vspace{\baselineskip}}
% \newenvironment{proof}{{\sl Proof\/}:\ \ }{\vspace{\baselineskip}}
 \newenvironment{points}{\hrule\vspace{0.5\baselineskip}\sfbegin These
 points should be
 covered:\begin{itemize}}{\end{itemize}\sfend\hrule\vspace{\baselineskip}}
 \newenvironment{example}{{\bf Example}.}{\vspace{\baselineskip}}

 % references
 \renewcommand{\eqref}[1]{(\ref{eq:#1})}
 \newcommand{\figref}[1]{Fig.~\ref{fig:#1}}
 \newcommand{\Figref}[1]{Figure~\ref{fig:#1}}
 \newcommand{\tabref}[1]{Table~\ref{tab:#1}}
 \newcommand{\secref}[1]{Section~\ref{sec:#1}}
 \newcommand{\chpref}[1]{Chapter~\ref{chp:#1}}
 \newcommand{\appref}[1]{Appendix~\ref{app:#1}}
 \newcommand{\lemref}[1]{Lemma~\ref{lem:#1}}
 \newcommand{\thmref}[1]{Theorem~\ref{thm:#1}}
 \newcommand{\corref}[1]{Corollary~\ref{cor:#1}}
 \newcommand{\conref}[1]{Conjecture~\ref{con:#1}}
 \newcommand{\clmref}[1]{Claim~\ref{clm:#1}}

 % statistical quantities
 \newcommand{\Rrr}{\vec{R}_{\vec{r},\vec{r}}}
 \newcommand{\SINR}{{\mathrm{SINR}}}
 \newcommand{\UMSE}{{\mathrm{UMSE}}}
 \newcommand{\cm}{_{\mathrm{c}}}
 \newcommand{\mse}{_{\mathrm{m}}}
 \newcommand{\Jcm}{J_{\mathrm{c}}}
 \newcommand{\Jmse}{J_{\mathrm{m}}}
 \newcommand{\Jumse}{J_{\mathrm{u}}}
 \newcommand{\Jumsebnd}[1]{\Jumse\text{\raisebox{-0.5mm}
                {$\bigl|_{#1}^{\max}$}}}
 \newcommand{\Eumse}{{\cal E}_{\mathrm{u}}}
 \newcommand{\Eumsebnd}{\Eumse\text{\raisebox{-0.5mm}{$\bigl|^{\max}$}}}

 % multi-source
 \newcommand{\of}[1]{^{\scriptscriptstyle (#1)}}
 \newcommand{\ofc}[1]{^{{\scriptscriptstyle (#1)}*}}
 \newcommand{\oft}[1]{^{{\scriptscriptstyle (#1)}t}}
 \newcommand{\ofH}[1]{^{{\scriptscriptstyle (#1)}H}}

 % CDMA stuff
 \newcommand{\J}{J^{\scriptscriptstyle (\nu)}}
 \newcommand{\Jhat}{\Hat{J}^{\scriptscriptstyle (\nu)}}
 \newcommand{\Jz}{J_{\mathrm{z}}^{\scriptscriptstyle (\nu)}}
 \newcommand{\Jzhat}{\Hat{J}_{\mathrm{z}}^{\scriptscriptstyle (\nu)}}
 \newcommand{\Jm}{J_{\mathrm{m}}^{\scriptscriptstyle (\nu)}}
 \newcommand{\Jmhat}{\Hat{J}_{\mathrm{m}}^{\scriptscriptstyle (\nu)}}
 \newcommand{\Ja}{\Hat{J}_{{a}}}
 \newcommand{\Ju}{\Hat{J}_{{u}}}
 \newcommand{\Jt}{\Hat{J}_{{t}}}
 \newcommand{\Jp}{\Hat{J}_{{p}}}

 % Equalizer stuff
 \newcommand{\ftilde}{\tilde{\vec{f}}} % Equalizer Weight Error Vector
 \newcommand{\Nmax}{N_{\max}}

 % MMSE stuff (by Adam Margetts)
 %\def\argmin{\mathop{\rm arg\,min}}
 \newcommand{\argmin}[1]{\mathop{\rm arg}\,\mathop{\rm min}_{#1}}
 \newcommand{\pbox}{\parbox}
 %\newcommand{\argmin}[1]{\text{arg}\:\:\:{\mathop\text{min}}_{#1}}
 \newcommand{\rvec}{\vec{r}}
 \newcommand{\f}{\vec{f}}
 \newcommand{\vpsi}{\bvec{\psi}}
 \newcommand{\valpha}{\bvec{\alpha}}
%------------------------------------------------------------------
%------------------------------------------------------------------
%------------------------------------------------------------------
%------------------------------------------------------------------

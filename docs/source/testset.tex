%
% This material is based upon work supported by the Defense Advanced Research
% Projects Agency under Air Force Contract No. FA8702-15-D-0001. Any opinions,
% findings, conclusions or recommendations expressed in this material are those
% of the author(s) and do not necessarily reflect the views of the Defense
% Advanced Research Projects Agency.
%
% © 2019 Massachusetts Institute of Technology.
%
%
% This program is free software; you can redistribute it and/or modify
% it under the terms of the GNU General Public License version 2 as
% published by the Free Software Foundation;
%
% THIS SOFTWARE IS PROVIDED BY THE COPYRIGHT HOLDERS AND CONTRIBUTORS "AS IS"
% AND ANY EXPRESS OR IMPLIED WARRANTIES, INCLUDING, BUT NOT LIMITED TO, THE
% IMPLIED WARRANTIES OF MERCHANTABILITY AND FITNESS FOR A PARTICULAR PURPOSE
% ARE DISCLAIMED. IN NO EVENT SHALL THE COPYRIGHT OWNER OR CONTRIBUTORS BE
% LIABLE FOR ANY DIRECT, INDIRECT, INCIDENTAL, SPECIAL, EXEMPLARY, OR
% CONSEQUENTIAL DAMAGES (INCLUDING, BUT NOT LIMITED TO, PROCUREMENT OF
% SUBSTITUTE GOODS OR SERVICES; LOSS OF USE, DATA, OR PROFITS; OR BUSINESS
% INTERRUPTION) HOWEVER CAUSED AND ON ANY THEORY OF LIABILITY, WHETHER IN
% CONTRACT, STRICT LIABILITY, OR TORT (INCLUDING NEGLIGENCE OR OTHERWISE)
% ARISING IN ANY WAY OUT OF THE USE OF THIS SOFTWARE, EVEN IF ADVISED OF THE
% POSSIBILITY OF SUCH DAMAGE.

\chapter{MIT Lincoln Lab Testset}

The MIT Lincoln Laboratory testset is a predefined set of
interference nodes, environments, and scenarios used to create more
realistic interference situations.  Three basic scenarios are
included in the testset package---one of each type: urban, suburban,
and rural. Interference nodes in these scenarios broadcast
pre-recorded radio signals such as analog television, GSM, microwave
oven, digital television, satellite phone, FM broadcast, and 802.11.

\section{Using the Testset}

\begin{figure}[h]
\centering
\includegraphics[width=5in]{"figs/Testset"}
\caption{Diagram of LLAMAComm using LL testset}
\label{fig:testSetEx}
\end{figure}

Figure \ref{fig:testSetEx} shows how the LL testset is a simple
extension of LLAMAComm.  Testset nodes are simply included in the
scenario just like any other node.  Code for the testset is
contained in the folder \verb+llamacomm/testset.LL+. The
interference reference files are distributed separately and should
be copied to the folder \verb+llamacomm/testset.LL/refFiles12p5/+.

To try the testset, start Matlab and change the current directory to
\verb+llamacomm/+.  Run \verb+StartLL.m+.  If you get an error, most
likely you haven't moved the interference reference files into the
appropriate folder (see the preceding paragraph for instructions).

\section{Test Setup Files}
The test setup files, \verb+Rural1Test()+, \verb+Suburban1Test()+, or
\verb+Urban1Test()+, are a shortcut to creating an environment of a given type
(either rural, suburban, or urban) and populating the node array with
pre-defined interference nodes.  For example, the pseudocode outline of the
test setup file \verb+Rural1Test.m+ is as follows.

\begin{verbatim}
      Define rural environment parameters
      Instantiate the environment object
      Set the default interference block length
      Create the interference nodes
      Return the environment object and the node object array
\end{verbatim}

The interference nodes are created by calling
\verb+testset.LL/interf/BroadcastNode.m+ with the proper parameters.
This makes it easy to manipulate parameters such as interference
type, location, center frequency, and transmit power. See Tables
\ref{tab:rural1test}-\ref{tab:urban1test} for a listing of the
interference parameters.  Note that the type is obfuscated to make
the true interference type unknown---Lincoln Laboratory maintains
the secret decoder ring.  The interference file for each node is
randomly chosen from the available files for the corresponding
interference type.  Users of Cogcom should find it is much easier to
manage the interference parameters in LLAMAComm.

If users want to modify the code, they are advised to copy and paste the
contents into a new folder in the \verb+/llamacomm/user/+ directory.  That way
future updates to the testset will not overwrite user code.  One simple
modification is to comment out the interference sources in the test setup file
that are outside the bands of interest.

% Insert a table with the interference parameters
%
\renewcommand\arraystretch{2.0}
\begin{table}[htp]
\begin{center}
\caption{Interference parameters for {\tt Rural1Test.m}.}
\begin{tabular}{|c|c|c|c|c|}
%\multicolumn{1}{c|}{$T_of_o$}&\multicolumn{1}{c|}{$A$}&\multicolumn{1}{c|}{$B$}&\multicolumn{1}{c}{$B/A$}\\
\hline
    Name        & Type           & Location (m)         & $f_c$ (Hz)  & Power (Watts)\\
\hline \hline
 Norm1\_60MHz&     norm&     [-30e3, 0,      455]&   60e6&       35e2\\
\hline
 Fred1\_90MHz&     fred&     [30e3,  0,      450]&   90e6&       1.7e3\\
\hline
 Norm1\_450MHz&    norm&     [0,     30e3,   450]&   450e6&      25e2\\
\hline
 Norm1\_620MHz&    norm&     [-25e3, 25e3,   455]&   620e6&      25e2\\
\hline
 Nola1\_740MHz&    nola&     [0,     -30e3,  455]&   740e6&      25e3\\
\hline
 Cecil1\_850MHz&   cecil&    [-10e3, 0,      30]&    850e6&      79\\
\hline
 Sophia1\_1600MHz& sophia&   [100,   0,      2]&     1600e6&     3\\
\hline
\end{tabular}
\label{tab:rural1test}
\end{center}
\end{table}
%
%
\renewcommand\arraystretch{2.0}
\begin{table}[htp]
\begin{center}
\caption{Interference parameters for {\tt Surburban1Test.m}.}
\begin{tabular}{|c|c|c|c|c|}
%\multicolumn{1}{c|}{$T_of_o$}&\multicolumn{1}{c|}{$A$}&\multicolumn{1}{c|}{$B$}&\multicolumn{1}{c}{$B/A$}\\
\hline
    Name        & Type           & Location (m)         & $f_c$ (Hz)  & Power (Watts)\\
\hline \hline
    Norm1\_60MHz & norm&     [-16e3, 0, 315]& 60e6&       10e2 \\
\hline
    Fred1\_90MHz&     fred&     [0,     -16e3,  320]&   90e6& 125e3\\
\hline
    Norm1\_450MHz&    norm&     [14e3,  14e3,   315]&   450e6& 30e2\\
\hline
    Nola1\_620MHz&    nola&     [16e3,  0,      313]&   620e6&250e3\\
\hline
    Ned1\_740MHz&     ned&      [0,     -16e3,  321]&   740e6&20e3\\
\hline
    Cecil1\_850MHz&   cecil&    [600,   0,      27]&    850e6&125\\
\hline
    Cecil1\_1870MHz&  cecil&    [0,     600,    34]&    1870e6& 251\\
\hline
    Isabelle1\_2495MHz&isabelle&[0,     50,     2]&     2495e6& 5\\
\hline
\end{tabular}
\label{tab:surburban1test}
\end{center}
\end{table}
%
%
\renewcommand\arraystretch{2.0}
\begin{table}[htp]
\begin{center}
\caption{Interference parameters for {\tt Urban1Test.m}.}
\begin{tabular}{|c|c|c|c|c|}
%\multicolumn{1}{c|}{$T_of_o$}&\multicolumn{1}{c|}{$A$}&\multicolumn{1}{c|}{$B$}&\multicolumn{1}{c}{$B/A$}\\
\hline
    Name        & Type           & Location (m)         & $f_c$ (Hz)  & Power (Watts)\\
\hline \hline
 Nola1\_60MHz&     nola&     [0,     -5e3,   313]&   60e6&       250e3\\
\hline
 Fred1\_90MHz&     fred&     [0,     5e3,    320]&   90e6&       125e3\\
\hline
 Annette1\_450MHz& annette&  [-5e3,  0,      313]&   450e6&      250e3\\
\hline
 Annette1\_620MHz& annette&  [-5e3,  0,      313]&   620e6&      250e3\\
\hline
 Ned1\_740MHz&     ned&      [5e3,   0,      321]&   740e6&      20e3\\
\hline
 Cecil1\_850MHz&   cecil&    [0,     100,    28]&    850e6&      63\\
\hline
 Cecil1\_1870MHz&  cecil&    [100,   0,      28]&    1870e6&     251\\
\hline
 Ike1\_2495MHz&     ike&      [50,    0,      10]&    2495e6&     1\\
\hline
 Ike1\_5800MHz&     ike&      [-50,   0,      10]&    5800e6&     1\\
\hline
\end{tabular}
\label{tab:urban1test}
\end{center}
\end{table}
%
%
% This material is based upon work supported by the Defense Advanced Research
% Projects Agency under Air Force Contract No. FA8702-15-D-0001. Any opinions,
% findings, conclusions or recommendations expressed in this material are those
% of the author(s) and do not necessarily reflect the views of the Defense
% Advanced Research Projects Agency.
%
% © 2019 Massachusetts Institute of Technology.
%
%
% This program is free software; you can redistribute it and/or modify
% it under the terms of the GNU General Public License version 2 as
% published by the Free Software Foundation;
%
% THIS SOFTWARE IS PROVIDED BY THE COPYRIGHT HOLDERS AND CONTRIBUTORS "AS IS"
% AND ANY EXPRESS OR IMPLIED WARRANTIES, INCLUDING, BUT NOT LIMITED TO, THE
% IMPLIED WARRANTIES OF MERCHANTABILITY AND FITNESS FOR A PARTICULAR PURPOSE
% ARE DISCLAIMED. IN NO EVENT SHALL THE COPYRIGHT OWNER OR CONTRIBUTORS BE
% LIABLE FOR ANY DIRECT, INDIRECT, INCIDENTAL, SPECIAL, EXEMPLARY, OR
% CONSEQUENTIAL DAMAGES (INCLUDING, BUT NOT LIMITED TO, PROCUREMENT OF
% SUBSTITUTE GOODS OR SERVICES; LOSS OF USE, DATA, OR PROFITS; OR BUSINESS
% INTERRUPTION) HOWEVER CAUSED AND ON ANY THEORY OF LIABILITY, WHETHER IN
% CONTRACT, STRICT LIABILITY, OR TORT (INCLUDING NEGLIGENCE OR OTHERWISE)
% ARISING IN ANY WAY OUT OF THE USE OF THIS SOFTWARE, EVEN IF ADVISED OF THE
% POSSIBILITY OF SUCH DAMAGE.
